\section{Diskussion}
\label{sec:disc}
Wir haben nun in Kapitel 3 tiefergehend die Möglichkeiten des QML im Status quo reflektiert. Vom Ziel der Quantenüberlegenheit mit Rechnern, bei denen mindestens 1000 Qubits in letztendlich Quantenregistern miteinander verschränkt werden, sind wir mit aktuellen Größen um die 65 Qubits noch sehr weit von entfernt.\\

Wir haben aber in Kapitel 2 auch gesehen, was für gewaltiges Potential das Thema QML in sich trägt. Dementsprechend lohnt es sich definitiv, weiterhin an Verbesserungen und neuen Kenntnissen zu forschen.\\
Hierbei ist großes Durchhaltevermögen gefragt. Die Quantenmechanik als Grundlage ist ein zähes Unterfangen. In Kapitel 2.1 haben wir in den theoretischen Grundlagen gesehen, wie schwer diese Art der Physik zu greifen ist. Folgerungen wie die Verschränkung oder die Heisenbergsche Unschärferelation sind sehr unintuitiv, doch ein (Haupt-)Argument der Ästhetik ist in Wissenschaften fehl am Platz.\\

Außer Acht lassen dürfen wir hierbei nicht, dass die QAML Algorithmen in der NISQ-Ära keineswegs schlecht sind. Bereits diese hybriden Algorithmen stellen, im Vergleich zu klassischen ML, eine klare Verbesserung dar.\\
Diesen Standpunkt haben wir in Kapitel 3.2 beschrieben.\\

Da diese Arbeit nur den Ansatz des Quantum Gate Models verfolgt und beschrieben hat, sei zuletzt noch angemerkt, dass es einen weiteren sehr interessanten Ansatz im Bereich des Quantum Computing gibt.\\
Es handelt sich dabei um das sogennante Quantum Annealing (QA). Der Grundgedanke hier ist heuristischer Natur zur Optimierung mithilfe von Quanten Computern - auch im Bereich des QML. Für mehr Details verweisen wir hier auf \cite{Annealing}.\\

Wir verfolgen in der Wissenschaft also bereits zwei sehr interessante Umsetzungen, um das Ziel Quantenüberlegenheit erreichen zu können. Ein logischer Schluss wäre also nun, auch nach weiteren Ansätzen zu forschen.\\
Es könnte sich durchaus ergeben, dass wir ein größeres Umsetzungs- als Hardwareproblem haben. Vielseitige Forschung ist hier deswegen sehr sinnvoll und wichtig.