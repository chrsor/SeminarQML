%Anmerkung, da Fehler beim Kompilieren: Zeichen wie \ oder & sind in LaTeX Commandchars, müssen also bei normaler Benutzung mit \( \) oder $$ escapet werden - Chris

%TO DO 
%Anwendungsfälle zitieren hier evtl aus 1611 Matrix + Fourier transformationen
%mehr zu unlösbaren/langen Berechnungen
%mehr Quantum Machine Learning
\section{Einleitung}
\label{sec:intro}
Machine Learning (ML) scheint im Zeitalter der Informationstechnologie eine der großen und vielversprechenden „Killer“ Applikationen zu sein. Schon heute reichen die Anwendungsgebiete von Verkehr $\&$ Mobilität, über Marketing, Recht und Verwaltung bis hin zum Gesundheitswesen \cite{opportunitieschallenges1}. ML wird für Bilderkennung, Stimmungsanalyse, Videoüberwachung und auch für die tägliche E-Mail Klassifizierung und Spam-Filterung verwendet. Entsprechende Algorithmen ermöglichen dabei durch Mustereingaben das Erlernen verschiedener Anwendungsfälle, anstatt einer vordefinierten Vorgangsanweisung \cite{quantummethodssupervised1}. \\

\\
Viele Aufgabengebiete sind auch mit heutigem Technologiestandard – mit immer steigender Rechenleistung der Computersysteme – nur mit sehr großem Zeit-, Rechen- und somit auch Energieaufwand lösbar, manche dieser Probleme sind sogar so komplex, dass die Berechnung hierfür entweder Jahre dauern würde, oder diese sogar komplett unlösbar sind \cite{fraunhoferbigdata}.\\
Quanten Computing bietet eine ganz andere Herangehensweise an die Berechnung solcher sehr kostenintensiven Probleme. Durch die echte Parallelität, die Quanten Computer bei der Berechnung versprechen, dürfte es möglich sein, mehrere Lösungsansätze und -wege gleichzeitig zu berechnen und somit wesentlich größere Datenbestände in einem kürzeren Zeitfenster abzuarbeiten.\\
Es ist zu erwarten, dass Quanten Computing sofern es diese Versprechungen tatsächlich erfüllt, eine Revolution in der Informationstechnik einleiten wird. Die Frage bleibt bloß, in welchen Bereichen sich diese neue Technik besonders gut einsetzen lässt und für welche sich klassische Computer weiterhin besser eignen. \\
\\
Im Rahmen dieser Arbeit soll nun die  Einbindung von Quanten Computern im Bereich des Machine Learnings genauer betrachtet werden, insbesondere die kurzfristige Umsetzung des Quantum Machine Learnings (QML) in den kommenden Jahren.\\
Da ML-Algorithmen allgemein zu den komplexeren und kostenintensiven Berechnungen gehören, könnte es sich hierbei um einen idealen Anwendungsbereich für Quanten Computer handeln. Wie kann dies also umgesetzt werden, und wird es in der nahen Zukunft bereits möglich sein solche Quanten Machine Learning Algorithmen zu realisieren?\\
In diesem Sinne sollen hier zunächst die Grundlagen des Quantum Gate Models, sowie die Chancen die sich daraus potentiell in Zukunft für ML-Algorithmen ergeben, betrachtet werden. \\
Da wir uns aber noch in der Anfangsphase der Entwicklung nutzbarer Quanten Computer befinden, soll im Folgenden auf die aktuelle Realität, zu überwindende Hürden und mögliche Lösungen dafür eingegangen werden. Insbesondere sollen hierbei Hybride Methoden, die auf einer Kombination aus klassischen und Quanten Computern aufbauen, betrachtet werden.\\
Insgesamt Soll diese Arbeit im Folgenden also einen Überblick schaffen über die möglichen, die bisherigen und zuletzt die in naher Zukunft erreichbaren technischen Fortschritte, die Quanten Computing für Machine Learning Algorithmen bietet.

