%TO DO
%ca. 1 Seite
%Ausblick schreiben
\section{Schluss}
\label{sec:conclusion}
Wir befinden uns in einer spannenden Phase der Menschheitsgeschichte, Wissenschaft und Technologie sind an einem Punkt angelangt, der vor wenigen Jahrzehnten noch als undenkbar galt. Schon bald werden wir die Auswirkungen der heutigen Forschungsarbeiten auf diesem Gebiet greifen können, die Riesen der Welt der Informationstechnologie arbeiten mit Hochdruck an dem Durchbruch der Quanten Computer.\\

Auch schon in den kurzfristig kommenden Jahren erwarten uns spannende Zeiten, die NISQ-Ära verspricht uns erste große Erfolge im Bereich des QAML. Hybride Methoden bieten auf kurze Sicht eine interessante Herangehensweise an Optimierungsprobleme. Natürlich bedeuten diese noch nicht, dass die digitale und reale Welt komplett auf den Kopf gestellt wird, allerdings ist zu erwarten, dass das Arbeiten mit diesen Methoden den Weg zu großen Durchbrüchen mit Quantum Computern und Quantum Algorithmen ebnen wird.\\

Die Zukunft des Machine Learnings wird vermutlich angeführt vom Wort Quantum sein, viele Schritte sind bis zu diesem Ziel nötig, große Meilensteine sind störfreie Quantengatter, und ganz besonders Quanten Fehlerkorrektur. Trotz aller Stolpersteine und ausstehenden Erfolge schauen wir mit großer
Begeisterung den Entdeckungen und Errungenschaften der nächsten Jahre entgegen, das Potential der Quanten Technologie ist gigantisch.\\

Kleine Quantencomputer und größere für Spezialanwendungen entwickelte
Quantensimulatoren, Annealer etc. können bereits vielversprechende Anwendungen im Bereich von Machine Learning und allgemeiner Datenanalyse durchführen. Allerdings benötigen die hierfür eingesetzten Algorithmen für eine Anwendung auf große Datensätze und reale Probleme viel mehr Rechenleistung \cite{biamonte2017quantum}.\\
Kurzum benötigen sie Quanten Hardware. Hierbei stellt sich die Frage: Kann dieses Ziel erreicht werden? \\ 

Große Fortschritte sind bisher schon zu verzeichnen, es gibt Möglichkeiten für Forscher aus aller Welt um mit Quanten Computern und deren Leistung in Berührung zu kommen und somit zu ermöglichen, dass bereits jetzt eine Vielzahl kluger Köpfe an Algorithmen und Lösungen für die Zukunft arbeiten können. Ganz im Sinne von Open-Source. Dies wird in der Zukunft weiter ausgebaut werden, Quantum Cloud Computing - also der Zugriff auf Quanten Computer über die Cloud - wird verfügbar gemacht werden und der allgemeine technologische Standard von Quanten Computern wird in dieser Zeit ebenso voranschreiten, was Größe und Komplexität betrifft \cite{biamonte2017quantum}.\\

Sollten die Herausforderungen der NISQ-Ära überwunden werden, so steht einer "Quantum Zukunft" nichts mehr im Wege. Besonders die Möglichkeit der Einbindung bzw. Kombination von Quantum Machine Learning und der Kontrolle von Quantum Computern stellt einen Schritt dar, der einen sehr zügigen Fortschritt ermöglichen kann.\\ So kann der Quanten Computer bzw. das Quanten Machine Learning einer Generation verwendet werden, um die der nächsten Generation zu erschaffen, was zu einem sich selbst verstärkenden Zyklus führen kann. \\

