% +------------------------------------------------------------------------------+ %
% | EDIT THESE SETTINGS ACCORDING TO YOUR THESIS                                 | %
% +------------------------------------------------------------------------------+ %

\newcommand{\seminar}{Trends in Mobilen und Verteilten Systemen}

\newcommand{\authorA}{Nadja Heinecke}
\newcommand{\authorB}{Christopher Sorg}
\newcommand{\authorC}{Daniel Seidl}
\newcommand{\supervisor}{Leo Alexander Sünkel}

\newcommand{\thesistitle}{Quantum Machine Learning - Ideal und Wirklichkeit}

\newcommand{\thesisabstract}{% \/ put your thesis abstract below \/
Diese Arbeit soll die Möglichkeiten der Nutzung von Quanten Computern für Machine Learning Algorithmen erforschen. Hierfür wollen wir zunächst ein paar Grundlagen des Quantum Computing für den Leser auffrischen, bevor wir auf Nutzungsbeispiele eingehen. Insbesondere soll aber auch die Anwendbarkeit auf in den nächsten Jahren realisierbaren, sogenannten Noisy Intermediate-Scale Quantum (NISQ) Computern beleuchtet werden. 
\\Wir möchten hiermit einen Überblick über lang- und kurzfristige Ziele des Quantum Machine Learnings und deren Umsetzbarkeit bieten.
}

\newcommand{\thesisauthorship}{% \/ describe who wrote what below \/
\authorB $\,$ hat die Kapitel \ref{sec:qgmc} und \ref{sec:disc} verfasst.\\
\authorC \ hat Kapitel \ref{sec:conclusion} und gemeinsam mit \authorA \ die Kapitel \ref{sec:intro} und \ref{sec:nisq} verfasst.
}


\selectlanguage{ngerman} 